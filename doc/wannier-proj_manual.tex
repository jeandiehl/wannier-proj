\documentclass[a4paper,bibtotocnumbered]{scrartcl}

\usepackage[utf8]{inputenc}
\usepackage{ngerman}
\usepackage{graphicx}
\usepackage{url}
\usepackage[pdftex]{hyperref}
\usepackage{booktabs}
\usepackage{tikz}
\usepackage[basic]{circ}
\usepackage{amsmath}
\usepackage{latexsym}
\usepackage{import}
\usepackage{subfig}
\usepackage{hyperref}

\begin{document}

\begin{center} 
\sffamily
\vspace*{5mm}

\Huge{Manual:}

\Huge{\texttt{wannier-proj}}
\vspace{7mm}

\normalsize{
Goethe-Universität, Frankfurt am Main \\
Institut für Theoretische Physik \\
Jean Diehl\\
August 2013}
\vspace{38mm}

\end{center}

\section{Introduction}
\texttt{wannier-proj} is an implementation to calculate projectors from bloch used in wien2k to wannier states as described in \cite{aichhorn}.

The latest version can be found in\\ \url{http://itp.uni-frankfurt.de/~jdiehl/download/wannier-proj/}.

\section{Installation and compilation}
\begin{itemize}
	\item \texttt{wannier-proj} uses the Eigen library for linear algebra arithmetics. You can get it from\\
	\url{http://eigen.tuxfamily.org/}
	\item unzip where you want, for example your home
	\item add the following to your \texttt{\textasciitilde /.bashrc}
	\begin{verbatim}
		export CPLUS_INCLUDE_PATH=~/eigen-3.1.3
	\end{verbatim}
	\item run from bash
	\begin{verbatim}
		source ~/.bashrc
	\end{verbatim}
	\item now the Eigen library should be available for the linker from this
	terminal without reboot. If not try rebooting.
	
	\item unzip the \texttt{wannier-proj} files you downloaded
	\item run change to wannier-proj directory and run \texttt{make}
	\item now \texttt{wannier-proj} should be compiled, check subdirectory \texttt{wannier-proj/bin}
	for existing executables
\end{itemize}

\section{Preparation}
\begin{itemize}
	\item You need a fully converged \texttt{wien2k} calculation
	\item change the \texttt{TOT} switch in the \texttt{case.in2} in your
	wien2k project directory to \texttt{ALM}
	\item this will produce the \texttt{case.almblm} file when running \texttt{x lapw2}
	\item use not the original \texttt{lapw2} (this will produce a really huge file),
	instead the modified one in \texttt{wannier-proj/wien2k/}
	\item create or use the \texttt{case.inproj} file in your wien2k project
	directory, rename it to your case
	\begin{verbatim}
		1		#  number of atoms
		2  1  2		#  atom index, No. of orbitals, l
		-5.5 3.0	#  Energy window
		45.0 0.0 0.0 1	#  Euler angles (zyz convention), rad/degree 

	\end{verbatim}
	\item If you want for example project out the Fe3d orbitals, then you set the first line
	to \texttt{1} for one (irreducible) iron atom, the second line to \texttt{2  1  2} referring
	with the first number to the iron atoms (index in \texttt{case.struct}, second
	number one orbital shell in total you project out, third number meaning
	you chose the $l=2$, i.e. d orbitals, to be projected. In the third line you
	select an energy window. Bigger is better, but try to narrow down. It is
	very sensitve to this window. At first do not care about the last line ;)
	\item this file acts as an input file for the modified \texttt{lapw2},
	selecting only certain orbitals in a defined energy window
	\item it also acts as an input file later for the \texttt{wannier-proj}, so if
	you play around with different orbitals and energy windows, select first a
	bigger range and run \texttt{lapw2}, later you can narrow down without
	reproducing the \texttt{case.almblm} file again, if you select in the later
	steps a subset of orbitals/energies you have written to this file
	\item run \texttt{x lapw2}
	\item copy \texttt{case.indm} from the wien2k \texttt{SRC\_templates} folder to
	your wien2k project directory
	\item this file is not important for the modified \texttt{lapwdm} from the
	\texttt{wannier-proj/wien2k} directory, but needs to be there. so no changes needed.
	\item run \texttt{x lapwdm}
	\item this procedure should have at least produced the following files
	\begin{verbatim}
		case.almblm
		case.rot
		case.symm
	\end{verbatim}
	\item additionally you should also have at least the following files from scf-cycle:
	\begin{verbatim}
		case.struct
		case.energy
		case.scf
	\end{verbatim}
	
\end{itemize}
\section{Running \texttt{wannier-proj}}
\begin{itemize}
	\item all the following executables can be found in \texttt{wannier-proj/bin}
	\item run \texttt{init\_smat} in your wien2k project folder to produce the \texttt{case.smat}
	file, which contains the connection between spherical and cubic harmonics meaning
	conversion to irreducible representation of orbitals (yet only s, p, d, no f orbitals)
	\item now you could - if you want - narrow down the seletion of orbitals or energies
	in the \texttt{case.inproj} file
	\item running now \texttt{run\_proj} will produce the projectors, and the following output files
	\begin{verbatim}
		case.projtilde
		case.overtilde
		case.proj
		case.over
		case.outputproj
	\end{verbatim}
	\item \texttt{case.projtilde} the auxiliary projectors no orthonormalized,
	\texttt{case.overtilde} the corresponding overlap matrix, \texttt{case.proj}
	the orthogonalized/-normalized projectors, and to check this \texttt{case.over}
	their overlap matrix - which should always be a unitary matrix for each kpoint.
	\item \texttt{case.outputproj} summarizes which orbitals where projected out,
	the first rows (also in this order) name the columns of the projector files. The
	later energy lines, give the energy index from the \texttt{case.energy} file
	which corresponds to the rows in the projector files.
\end{itemize}

\newpage
\begin{thebibliography}{99}
\bibitem{aichhorn}
Phys Rev B 80, 085101 (2009)


\end{thebibliography}

\end{document}
